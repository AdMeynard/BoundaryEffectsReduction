\documentclass[11pt,letterpaper,roman]{moderncv}

\usepackage[english]{babel}
\usepackage{ragged2e}
\usepackage{graphicx} 
\usepackage{xcolor} 

\moderncvstyle{classic}
\moderncvcolor{blue} 
\nopagenumbers{}
\usepackage[utf8]{inputenc}
\usepackage[scale=0.75]{geometry}
\usepackage{footmisc}
%\usepackage[style=mla]{biblatex}

\name{Adrien Meynard\\}{Hau-Tieng Wu}
\title{Cover letter}
\address{Duke University}{Durham, NC}{USA}
\email{adrien.meynard@duke.edu}           
  
%\bibliography{Sampta17}

\begin{document}
\recipient{Prof. Lorenzo Galleani}{IEEE Transactions on Signal Processing}
\date{January 29, 2021}
\opening{Dear Associate Editor,}
\closing{Yours faithfully,}

\makelettertitle
\justify
 
Please find the revised manuscript:
\begin{center}
\emph{An Efficient Forecasting Approach to Reduce Boundary Effects in Real-Time Time-Frequency Analysis}
\end{center}
which we are submitting for possible publication in \textit{IEEE Transactions on Signal Processing}.

Thank you for the first evaluation of our article. In this new version, we have taken into account the reviewers' comments and suggestions as much as we could. In particular, we would like to mention that we have addressed the points that you emphasized in the decision letter. Here are our responses.

\begin{enumerate}
\item
\textit{In section IV.C-2, discuss how the performances of the proposed method change with respect to $M$, and explain how to tune its value in applications (comment 3 by Reviewer 1).}\\
\textbf{We have added the new paragraph IV-C2a where we discuss the influence of the parameter $M$ on the forecasting performance of an adaptive harmonic signal. In particular, we mention that choosing a too small $M$ does not provide enough information to forecast the signal satisfactorily, while a too large $M$ makes the algorithm insensitive to local nonstationarities produced by frequency variations. The user must then make a compromise on the choice of $M$ in order to optimize the performance of the algorithm. A choice of $M$ such that the subsignals contain about a dozen oscillations is generally appropriate.}
\item
\textit{Discuss the shadowing approach proposed by C. Gregobi and compare it to your method (comment 1 by Reviewer 2).}\\
\textbf{We have mentioned this approach in the introduction of our article. We precise that dynamic mode predictors have been proposed in speech processing by Vargas and McLaughlin, to forecast signals  falling into the so-called source--filter model. This gradient-based technique rely on the shadowing approach proposed in Grebogi et al. Nevertheless,  since the source--filter is not adapted to model physiological signals on which we focus, we do not implement this method in our article.}
\item
\textit{Discuss the robustness of your method with respect to possible stochastic components in the signal (comment 2 by Reviewer 2).}\\
\textbf{A new paragraph has been added to precise the limitations of the theoretical justifications on the performance of \textsf{SigExt}. Indeed, the models of sections III-A and III-C consider the meaningful part of the signal to be a deterministic component. These models are purposely adapted to signals showing local line spectra. The physiological signals we are interested in, such as respiratory or cardiac signals, have this characteristic (see section IV-D). Signals with wider local spectra, such as electroencephalography signals, do not fall into this category, and are more faithfully modeled as random signals. Thus, the theoretical justifications proposed above are no longer applicable to guarantee the forecasting quality of \textsf{SigExt}. A more robust approach to the study of signals following this model is envisaged in our future work.}
\end{enumerate}

A document containing detailed responses to the reviewers' comments is enclosed to this letter.

\makeletterclosing

\end{document}