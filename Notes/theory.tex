\subsection{General signal model}
We model the observed signal as a given waveform corrupted by an additive Gaussian white noise. Therefore, the measured discrete signal $\bx$ is written as:
\[ 
\bx = \bz + \sigma\bw\ ,
\]
where $\bz$ is a deterministic signal, $\bw$ is a (Gaussian) white noise, whose variance is normalized to one. Thus, $\sigma^2$ denotes the variance of the additive noise $\sigma\bw$.

\subsection{Forecasting error}
On the forecasting interval, we decompose the estimated signal $\tilde\bx$ as follows:
\[
\tilde \bx[N-1+\ell] = \bz[N-1+\ell] + \bepsilon[\ell]\ ,
\]
where $\bepsilon\in\RR^L$ is a the forward forecasting error, which would ideally behave like the measurement noise $\sigma\bw$.  To evaluate the actual behavior of the forward forecasting error $\bepsilon$, we determine its two first moments.
\begin{enumerate}
\item 
The mean, which is also the estimation bias, is such that: 
\begin{align}
\nonumber
\bmu[\ell]&\defeq\EE\{\bepsilon[\ell]\} \\
\nonumber
&= \EE\{\be_M^T\tilde\bA^\ell\}\bz_{K} + \EE\{\be_M^T\tilde\bA^\ell\bw_{K}\} - \bz[N-1+\ell] \\
&=  \EE\{\balpha^{(\ell)}\}\bz_{K} + \sigma\EE\{\balpha^{(\ell)}\bw_{K}\} - \bz[N-1+\ell]
\label{eq:bias.1}
\end{align}
where $\balpha^{(\ell)}\in\RR^{1\times M}$ denotes the last row of $\tilde\bA^\ell$.

\item 
The covariance is given by:
\begin{equation*}
\bgamma[\ell,\lambda] \defeq \EE\{\left(\bepsilon[\ell]-\bmu[\ell]\right)\left(\bepsilon[\lambda]-\bmu[\lambda]\right)\} 
\end{equation*}
By means of the Cauchy-Schwarz inequality the covariance, we provide an upper bound on the covariance:  
\begin{equation}
\left|\bgamma[\ell,\lambda]\right| \leq \sqrt{\bgamma[\ell,\ell]\bgamma[\lambda,\lambda]}\ .
\end{equation}
We then focus on the determination of $\bgamma[\ell,\ell]$. Thus:
\begin{align*}  
\bgamma[\ell,\ell] &= \EE\{\bepsilon[\ell]^2\} - \bmu[\ell]^2 \\
&=\bz[N-1+\ell]^2 - 2\bz[N-1+\ell] \EE\{\tilde\bx[N-1+\ell]\} + \EE\{\tilde\bx[N-1+\ell]^2\} - \bmu[\ell]^2 \ .
\end{align*}
But:
\begin{align*}
\EE\{\tilde\bx[N-1+\ell]^2\} &= \EE\{(\balpha^{(\ell)}\bz_K)^2\} + 2\sigma\EE\{(\balpha^{(\ell)}\bz_K)(\balpha^{(\ell)}\bw_K)\} + \sigma^2\EE\{(\balpha^{(\ell)}\bw_K)^2\}
\end{align*}
%\EE\{\bx[N-1+\ell]^2\} &= \bz[N-1+\ell]^2 +\sigma^2 \\
%\EE\{\bx[N-1+\ell]\tilde\bx[N-1+\ell]\} &= \bz[N-1+\ell]\EE\{\tilde\bx[N-1+\ell]\} + \sigma\underbrace{\EE\{\bw[N-1+\ell]\tilde\bx[N-1+\ell]\}}_{=\EE\{\bw[N-1+\ell]\}\EE\{\tilde\bx[N-1+\ell]\}=0} \\
%&= \bz[N-1+\ell]\left(\EE\{\balpha^{(\ell)}\}\bz_{K} + \sigma\EE\{\balpha^{(\ell)}\bw_{K}\}\right) \\
%&= \bz[N-1+\ell]^2 + \bz[N-1+\ell]\bmu[\ell]\\
The mean square error can therefore be written as follows:
\begin{align}
\nonumber
\bgamma[\ell,\ell] =&-\bz[N-1+\ell]^2 -2\bz[N-1+\ell]\bmu[\ell] +\EE\{(\balpha^{(\ell)}\bz_K)^2\}  \\
&+ 2\sigma\EE\{(\balpha^{(\ell)}\bz_K)(\balpha^{(\ell)}\bw_K)\} + \sigma^2\EE\{(\balpha^{(\ell)}\bw_K)^2\} - \bmu[\ell]^2\ .
\end{align}
\end{enumerate}

\paragraph{Decomposition of $\balpha^{(\ell)}$.}
For all $m,m'\in\{0,\ldots,M-1\}$, we have:
\begin{align}
(\bX\bX^T)[m,m'] &= \sum_{k=0}^{K-1} \bx[\underbrace{N-K-M}_{\defeq N_0}+m+k]\bx[N-K-M+m'+k] = K(\bS^{(0)} + \bE^{(0)}) \\
(\bY\bX^T)[m,m'] &= \sum_{k=0}^{K-1} \bx[N_0+m+1+k]\bx[N_0+m'+k] = K(\bS^{(1)} + \bE^{(1)})\ ,
\end{align}
where
\[
\bS^{(a)}[m,m'] = \sigma^2\delta_{(m+a)m'}+\dfrac1{K}\sum_{k=0}^{K-1} \bz[N_0+m+a+k]\bz[N_0+m'+k] \ ,
\]
and $\bE^{(a)} = \sigma\bE_1^{(a)} + \sigma^2\bE_2^{(a)}$ with:
\[
\bE_1^{(a)}[m,m'] = \dfrac1K\sum_{k=0}^{K-1} \bz[N_0+m+a+k]\bw[N_0+m'+k] + \bw[N_0+m+a+k]\bz[N_0+m'+k]\ ,
\]
and
\[
\bE_2^{(a)}[m,m'] =  \dfrac1K\sum_{k=0}^{K-1} \bw[N_0+m+a+k]\bw[N_0+m'+k] - \delta_{(m+a)m'}\ ,
\]
with $a\in\{0,1\}$.
\begin{remark}
The matrices $\bE^{(0)}$ and $\bE^{(1)}$ are said to be error matrices because:
\begin{align*}
\EE\{\bE^{(0)}\} &= \EE\{\bE_1^{(0)}\} = \EE\{\bE_2^{(0)}\} = \bzero \\
\EE\{\bE^{(1)}\} &= \EE\{\bE_1^{(1)}\} = \EE\{\bE_2^{(1)}\} = \bzero\ .
\end{align*}
\end{remark}
Thus:
\begin{align*}
\tilde\bA &= (\bS^{(1)}+\bE^{(1)})(\bS^{(0)}+\bE^{(0)})^\inv\ .
\end{align*}
Let $\tilde\bA_0=\bS^{(1)}{\bS^{(0)}}^\inv$. Then:
\begin{align*}
\balpha^{(\ell)} &= \be_M^T\tilde\bA^\ell \\
&= \balpha^{(\ell)}_0 + \bh^{(\ell)}\ ,
\end{align*}
where
\begin{align}
\nonumber
\balpha^{(\ell)}_0 &= \be_M^T\tilde\bA_0^\ell \\
\bh^{(\ell)} &= \be_M^T\left(\left((\bS^{(1)}+\bE^{(1)})(\bS^{(0)}+\bE^{(0)})^\inv\right)^\ell- \tilde\bA_0^\ell\right)
\label{eq:error.vec}
\end{align}
%Because $\sigma$ is small, we develop the error vector $\bh^{(\ell)}$ using a second-order Taylor expansion with respect to $\sigma$. Then, a development  of the expression~\eqref{eq:error.vec} gives:
%\begin{align*}
%\bh^{(\ell)} &= \be_M^T\left( \left( \left(\bS^{(1)}{\bS^{(0)}}^\inv+\bE^{(1)}{\bS^{(0)}}^\inv\right)\left(\bI+\sigma(\bE_1^{(0)}+\sigma\bE_2^{(0)}){\bS^{(0)}}^\inv\right)^\inv \right)^\ell- \tilde\bA_0^\ell\right) \\
%& = \be_M^T\left( \left( \left(\bS^{(1)}{\bS^{(0)}}^\inv+\bE^{(1)}{\bS^{(0)}}^\inv\right)\left(\bI-\sigma(\bE_1^{(0)}+\sigma\bE_2^{(0)}){\bS^{(0)}}^\inv + \sigma^2\bE_1^{(0)}{\bS^{(0)}}^{-1}\bE_1^{(0)}{\bS^{(0)}}^{-1} +\bo(\sigma^2)\right)\right)^\ell- \tilde\bA_0^\ell\right) \\
%& = \be_M^T\bigg( \Big( \bS^{(1)}{\bS^{(0)}}^\inv+\sigma\bE^{(1)}_1{\bS^{(0)}}^\inv+\sigma^2\bE^{(1)}_2{\bS^{(0)}}^\inv - \sigma \bS^{(1)}{\bS^{(0)}}^\inv\bE_1^{(0)}{\bS^{(0)}}^\inv - \sigma^2\bS^{(1)}{\bS^{(0)}}^\inv\bE_2^{(0)}{\bS^{(0)}}^\inv\\
%& \hspace{50pt}-\sigma^2\bE^{(1)}_1{\bS^{(0)}}^{-1}\bE_1^{(0)}{\bS^{(0)}}^{-1} + \sigma^2\bS^{(1)}{\bS^{(0)}}^\inv\bE_1{\bS^{(0)}}^{-1}\bE_1^{(0)}{\bS^{(0)}}^{-1} +\bo(\sigma^2)\Big)^\ell- \tilde\bA_0^\ell\bigg) \\
%&= \be_M^T\left( \tilde\bA_0^\ell\left( \bI+\sigma\tilde\bA_0^{-1}\bQ_1+\sigma^2\tilde\bA_0^{-1}\bQ_2 +\bo(\sigma^2)\right)^\ell- \tilde\bA_0^\ell\right)\ ,\\
%\end{align*}
%where:
%\begin{align*}
%\bQ_1 &= \left( \bE^{(1)}_1- \tilde\bA_0\bE_1^{(0)} \right){\bS^{(0)}}^\inv\\
%\bQ_2 &= \left( \bE^{(1)}_2 - \tilde\bA_0\bE_2^{(0)} - \bE^{(1)}_1{\bS^{(0)}}^{-1}\bE_1^{(0)} + \tilde\bA_0\bE_1^{(0)}{\bS^{(0)}}^{-1}\bE_1^{(0)}\right){\bS^{(0)}}^\inv \ .
%\end{align*}
%Then:
%\begin{align*}
%\bh^{(\ell)} &= \balpha_0^{(\ell-1)}\left(\ell\sigma\bQ_1 + \ell\sigma^2\bQ_2 + \sigma^2\dfrac{\ell(\ell-1)}{2}\bQ_1\tilde\bA_0^{-1}\bQ_1 + \bo(\sigma^2)\right)\ .
%\end{align*}
Furthermore, from equation~\eqref{eq:bias.1} we have:
\begin{equation*}
\bmu[\ell] = \balpha_0^{(\ell)}\bz_{K} + \EE\{\bh^{(\ell)}\}\bz_{K} + \sigma\EE\{\bh^{(\ell)}\bw_{K}\} - \bz[N-1+\ell]
\end{equation*}
%with:
%\begin{align*}
%\EE\left\{\bh^{(\ell)}\right\} &= \ell\sigma^2\balpha_0^{(\ell-1)}\left(\EE\{\bQ_2\} + \dfrac{(\ell-1)}{2}\EE\{\bQ_1\tilde\bA_0^{-1}\bQ_1\}\right)  + \bo(\sigma^2)\\
%\sigma\EE\left\{\bh^{(\ell)}\bw_K\right\} &= \ell\sigma^2\balpha_0^{(\ell-1)}\EE\{\bQ_1\bw_K\} + \bo(\sigma^2)
%\end{align*}
%Then:
%\begin{align*}
%\be_M^T\EE\left\{\bQ_2\right\} =& -\EE\left\{ \bq_1{\bS^{(0)}}^{-1}\bE_1^{(0)} \right\}{\bS^{(1)}}^\inv\\
%\be_M^T\EE\left\{\bQ_1^2\right\} =& \left( \EE\left\{\bq_1{\bS^{(1)}}^{-1}\bE^{(1)}_1\right\} - \EE\left\{\bq_1{\bS^{(0)}}^{-1}\bE_1^{(0)}\right\} \right) {\bS^{(1)}}^{-1} \\
%\be_M^T\EE\{\bQ_1\tilde\bA_0^\ell\bw_K\} =& \EE\left\{\bq_1{\bS^{(1)}}^{-1}\tilde\bA_0^\ell\bw_K\right\}
%\end{align*}
%Then:
%\begin{equation}
%\bmu[\ell] =  \balpha_0^{(\ell)}\bz_{K} - \bz[N-1+\ell] + \ell\sigma^2\balpha_0^{(\ell-1)}\bu + o(\sigma^2)
%\label{eq:bias.2}
%\end{equation}
%where
%\begin{equation}
%\bu = \EE\left\{\bQ_1\bw_K\right\}-\left(\EE\{\bQ_2\} + \dfrac{(\ell-1)}{2}\EE\{\bQ_1\tilde\bA_0^{-1}\bQ_1\}\right)\bz_K \ .
%\label{eq:bias.err}
%\end{equation}
Besides
\begin{align*}
\bgamma[\ell,\ell] =& -\bz[N-1+\ell]^2 -2\bz[N-1+\ell]\bmu[\ell] +(\balpha_0^{(\ell)}\bz_K)^2 +2\balpha_0^{(\ell)}\bz_K\EE\{\bh^{(\ell)}\}\bz_K+ \EE\{(\bh^{(\ell)}\bz_K)^2\}  \\
&+ 2\sigma\balpha_0^{(\ell)}\bz_K\EE\{\bh^{(\ell)}\bw_K\} + 2\sigma\EE\{\bh^{(\ell)}\bz_K\bh^{(\ell)}\bw_K\} + 2\sigma\EE\{\bh^{(\ell)}\bz_K\balpha_0^{(\ell)}\bw_K\} + \sigma^2\EE\{(\bh^{(\ell)}\bw_K)^2\} \\
&+ 2\sigma^2\balpha_0^{(\ell)}\EE\{\bw_K\bh^{(\ell)}\bw_K\} + \sigma^2\|\balpha_0^{(\ell)}\|^2  - \bmu[\ell]^2\\
=&\ 2\balpha_0^{(\ell)}\bz_K\bz[N-1+\ell] -(\balpha_0^{(\ell)}\bz_K)^2-\bz[N-1+\ell]^2 +2\left(\balpha_0^{(\ell)}\bz_K-\bz[N-1+\ell]\right)\bmu[\ell] \\
& +\EE\{(\bh^{(\ell)}\bz_K)^2\} + 2\sigma\EE\{\bh^{(\ell)}\bz_K\bh^{(\ell)}\bw_K\} + 2\sigma\EE\{\bh^{(\ell)}\bz_K\balpha_0^{(\ell)}\bw_K\} + \sigma^2\EE\{(\bh^{(\ell)}\bw_K)^2\} \\
&+ 2\sigma^2\balpha_0^{(\ell)}\EE\{\bw_K\bh^{(\ell)}\bw_K\} + \sigma^2\|\balpha_0^{(\ell)}\|^2 - \bmu[\ell]^2\ .
\end{align*}
%Therefore:
%\begin{equation}
%\bgamma[\ell,\ell] = \sigma^2\bv[\ell] + o(\sigma^2)
%\label{eq:mse}
%\end{equation}
%where
%\begin{equation}
%\bv[\ell] = \|\balpha_0^{(\ell)}\|^2 +2\ell\balpha_0^{(\ell)}\EE\{\bw_K \balpha_0^{(\ell-1)}\bQ_1\}\bz_K +\ell^2\EE\{(\balpha_0^{(\ell-1)}\bQ_1\bz_K)^2\}
%\label{eq:mse.err}
%\end{equation}

\subsection{Sum of sine waves}

In that section, the deterministic part of the observed signal is assumed to be a multicomponent harmonic signal, that is a sum of sine waves. Then:
\begin{equation}
\bz[n]=\sum_{j=0}^J\Omega_j\cos\left(2\pi f_j \frac{n}{\fs}\right)\ ,
\label{eq:sum.sine}
\end{equation}
where $J$ denotes the number of components, $\Omega_j$ the amplitude of the $j$-th component, and $f_j$ its frequency.

{\bf Assumption}: $f_j = \dfrac{p_j}{M}\fs = \dfrac{p'_j}{K}\fs$ for some $p_j$, $p'_j\in\NN^*$.

\paragraph{Evaluation of $\tilde\bA_0$ and $\balpha_0^{(\ell)}\bz_K$.}
\begin{align*}
\bS^{(a)}[m,m'] &= \sigma^2\delta_{(m+a)m'}+\sum_{j,j'=1}^J\dfrac{\Omega_j\Omega_{j'}}{K}\sum_{k=0}^{K-1} \cos\left(2\pi \frac{f_j}{\fs}(N_0+m+a+k)\right)\cos\left(2\pi \frac{f_{j'}}{\fs}(N_0+m'+k)\right) \\
&= \sigma^2\delta_{(m+a)m'}+\sum_{j=1}^J\dfrac{\Omega_j^2}{2K}\sum_{k=0}^{K-1} \cos\left(2\pi \frac{f_j}{\fs}(m+a-m')\right) + \cos\left(2\pi \frac{f_j}{\fs}(2k+m+a+m'+2N_0)\right)\\
&= \sigma^2\delta_{(m+a)m'}+\sum_{j=1}^J\left( \dfrac{\Omega_j^2}2\cos\left(2\pi \frac{f_j}{\fs}(m+a-m')\right) + \dfrac{\Omega_j^2}{2K}\underbrace{\sum_{k=0}^{K-1}\cos\left(2\pi \frac{f_j}{\fs}(2k+m+a+m'+2N_0)\right)}_{=0\ \mathrm{because}\ \frac{f_j}\fs=\frac{p'_j}K } \right)\\
&= \sigma^2\delta_{(m+a)m'}+\sum_{j=1}^J\dfrac{\Omega_j^2}2\cos\left(2\pi \frac{f_j}{\fs}(m+a-m')\right)\ .
\end{align*}
Thus, $\bS^{(0)}$ is a circulant matrix and is therefore diagonalizable in the Fourier basis:
\[
\bS^{(0)} = \bU\bLambda^{(0)}\bU^*\ ,
\]
where $\bU[m,m']=\frac1{\sqrt{M}}e^{-2\ii\pi mm'/M}$ and $\bLambda^{(0)} = \mathrm{diag}(\lambda_0^{(0)},\dots,\lambda_{M-1}^{(0)})$ with:
\begin{align*}
\lambda_m^{(0)} &= \sigma^2 + \sum_{j=1}^J\dfrac{\Omega_j^2}2\sum_{q=0}^{M-1} \cos\left(2\pi\frac{f_j}{\fs} q\right) e^{-2\ii\pi qm/M} \\
&= \sigma^2 + \dfrac{M}{4}\sum_{j=1}^J\Omega_j^2(\delta_{m,p_j} + \delta_{m,M-p_j})\ .
\end{align*}
Therefore:
\begin{align*}
{\bS^{(0)}}^\inv  &= \bU{\bLambda^{(0)}}^\inv\bU^*
\end{align*}
which leads to:
\begin{align*}
{\bS^{(0)}}^\inv[m,m']  &= \dfrac1{\sigma^2}\delta_{m,m'}-\sum_{j=1}^J\dfrac{\Omega_j^2}{2\sigma^2(\sigma^2+\Omega_j^2M/4)}\cos\left(2\pi p_j \dfrac{m-m'}{M}\right)\ ,
\end{align*}
and, consequently:
\begin{align}
\nonumber
\tilde\bA_0[m,m']  &= \sum_{q=0}^{M-1} {\bS^{(1)}}[m,q]{\bS^{(0)}}^\inv[q,m'] \\
&= \delta_{m+1,m'} + \sum_{j=1}^J\dfrac{2\Omega_j^2}{\Omega_j^2M+4\sigma^2}\cos\left(2\pi p_j\frac{m'}{M}\right)\delta_{m+1,M}
\label{eq:A0.sine}
\end{align}
Thus:
\begin{align*}
\tilde\balpha_0^{(1)}[m]  &=\sum_{j=1}^J\dfrac{2\Omega_j^2}{\Omega_j^2M+4\sigma^2}\cos\left(2\pi p_j\frac{m}{M}\right) \\
&= \dfrac{2}{M}\sum_{j=1}^J\cos\left(2\pi p_j\frac{m}{M}\right) + o(\sigma^2)\ .
\end{align*}
Besides, from equation~\eqref{eq:A0.sine}, we have
\begin{align*}
\tilde\bA_0\bz_{K} &= 
\begin{pmatrix}
\bz[N-M+1] \\
\vdots \\
\bz[N-1] \\
\balpha_0^{(1)}\bz_K
\end{pmatrix}
\end{align*}
By induction, we have:
\begin{align*}
\tilde\bA_0^{\ell}\bz_{K} &= 
\begin{pmatrix}
\bz[N-M+\ell] \\
\vdots \\
\bz[N-1] \\
\balpha_0^{(1)}\bz_K \\
\vdots \\
\balpha_0^{(\ell)}\bz_K
\end{pmatrix}
\end{align*}
Then:
\begin{align}
\nonumber
\balpha_0^{(\ell)}\bz_{K} &= \tilde\balpha_0^{(1)}\tilde\bA_0^{\ell-1}\bz_{K} \\
&=\sum_{m=0}^{M-\ell}\balpha_0^{(1)}[m]\bz[N-M+\ell+m-1]+\sum_{m=M-\ell+1}^{M-1}\balpha_0^{(1)}[m]\balpha_0^{(m-M+\ell)}\bz_K
\label{eq:seq}
\end{align}
But:
\begin{align*}
\balpha_0^{(1)}\bz_{K} &=\sum_{m=0}^{M-1}\balpha_0^{(1)}[m]\bz[N-M+m] \\
&=\sum_{j,j'=1}^J\Omega_{j'}\dfrac{2}{M}\underbrace{\sum_{m=0}^{M-1}\cos\left(2\pi p_j\frac{m}{M}\right)\cos\left(2\pi p_{j'}\dfrac{N+m}{M}\right)}_{=\delta_{j,j'}\frac2M\cos\left(2\pi p_j\frac{N}M\right)} + o(\sigma^2) \\
&= \sum_{j=1}^J\Omega_j\cos\left(2\pi p_j\dfrac{N}{M}\right) + o(\sigma^2) \\
&= \bz[N] + o(\sigma^2)
\end{align*}
and, by induction from~\eqref{eq:seq}:
\begin{equation}
\label{eq:alpha0z.sine}
\balpha_0^{(\ell)}\bz_{K} = \bz[N-1+\ell] + o(\sigma^2)
\end{equation}


%\paragraph{Evaluations of $\bu$ and $\bv$.}
%Because $\EE\{\bQ_2\} =  - \EE\{\bQ_1\bE^{(0)}_1\}{\bS^{(0)}}^\inv$, we only need derive $\bQ_1$ in order to determine $\bu$ and $\bv$ (see equations~\eqref{eq:bias.err} and~\eqref{eq:mse.err}). First, from equation~\eqref{eq:A0.sine} we have:
%\begin{align*}
%\bQ_1 &= \left( \bE^{(1)}_1- \tilde\bA_0\bE_1^{(0)} \right){\bS^{(0)}}^\inv \\
%&= 
%\begin{pmatrix}
%0 \\
%\vdots \\
%0 \\
%\bq{\bS^{(0)}}^\inv
%\end{pmatrix}
%\end{align*}
%where $\bq = \be_M^T\bE^{(1)}_1- \balpha^{(1)}_0\bE_1^{(0)}$. Then $\bq = \sum_{j=1}^J\bq_{j}$ with:
%\begin{align*}
%\bq_j[m] &= \dfrac{\Omega_j}K\sum_{k=0}^{K-1} \cos\left(2\pi p_j\dfrac{N-K+k}{M} \right)\bw[N_0+m+k] + \bw[N-K+k]\cos\left(2\pi p_j\dfrac{N_0+m+k}{M} \right) \\
%& \hspace{-50pt}- \sum_{q=0}^{M-1}\dfrac{2}{M}\cos\left(2\pi p_j\dfrac{q}{M}\right)\dfrac{\Omega_j}K\sum_{k=0}^{K-1} \cos\left(2\pi p_j\dfrac{N-K+q+k}{M} \right)\bw[N_0+m+k] + \bw[N_0+q+k]\cos\left(2\pi p_j\dfrac{N-K+m+k}{M} \right) \\
%&= \dfrac{\Omega_j}K\sum_{k=0}^{K-1} \left( \cos\left(2\pi p_j\dfrac{N-K+k}{M} \right) - \dfrac{2}{M}\underbrace{\sum_{q=0}^{M-1}\cos\left(2\pi p_j\dfrac{q}{M}\right) \cos\left(2\pi p_j\dfrac{N-K+q+k}{M} \right)}_{=\frac{M}{2}\cos\left(2\pi p_j \frac{M-K+k}{M}\right)} \right)\bw[N_0+m+k] \\
%& +\dfrac{\Omega_j}K\sum_{k=0}^{K-1} \underbrace{\left(\bw[N-K+k]-\dfrac{2}{M}\sum_{q=0}^{M-1}\cos\left(2\pi p_j\dfrac{q}{M}\right)\bw[N_0+q+k]\right)}_{\defeq\Delta_j[k]} \cos\left(2\pi p_j\dfrac{N_0+m+k}{M} \right) \\
%&= \dfrac{\Omega_j}K\sum_{k=0}^{K-1} \Delta_j[k] \cos\left(2\pi p_j\dfrac{N_0+m+k}{M} \right)
%\end{align*}
%Therefore:
%\begin{align*}
%\bq_j{\bS^{(0)}}^\inv[m] &= \sum_{q=0}^{M-1} \bq_j[q]{\bS^{(0)}}^\inv[q,m] \\
%&= \dfrac1{\sigma^2}\bq_j[m]-\sum_{j'=1}^J\dfrac{\Omega_{j'}^2}{2\sigma^2(\sigma^2+\Omega_{j'}^2M/4)}\sum_{q=0}^{M-1} \bq_j[q]\cos\left(2\pi p_{j'}\dfrac{q-m}{M}\right) \\
%&= \dfrac1{\sigma^2}\bq_j[m]-\dfrac{\Omega_j}{K}\sum_{k=0}^{K-1}\Delta_j[k]\sum_{j'=1}^J\dfrac{\Omega_{j'}^2}{2\sigma^2(\sigma^2+\Omega_{j'}^2M/4)} \underbrace{\left(\sum_{q=0}^{M-1} \cos\left(2\pi p_{j'}\dfrac{N_0+q+k}{M}\right)\cos\left(2\pi p_j\dfrac{q-m}{M}\right) \right)}_{=\frac{M}2\delta_{j,j'}\cos\left(2\pi p_j\dfrac{N_0+m+k}{M}\right)} \\
%&= \dfrac1{\sigma^2}\bq_j[m]-\dfrac{\Omega_{j}^2M}{\sigma^2(4\sigma^2+\Omega_{j}^2M)}\bq_j[m] \\
%&= \dfrac{1}{\sigma^2+\Omega_{j}^2M/4}\bq_j[m]\\
%&= \dfrac{4}{\Omega_{j}^2M}\bq_j[m] + o(\sigma^2)\ .
%\end{align*}

%Thus $\bQ_1=\bo(\sigma^2)$ and $\EE\{\bQ_2\}=\bo(\sigma^2)$. Consequently:
%\begin{align}
%\label{eq:v.sine}
%\bu &= \bo(\sigma^2) \\
%\label{eq:w.sine}
%\bv &= \|\balpha_0^{(\ell)}\|^2+\bo(\sigma^2)
%\end{align}

%\paragraph{Back to the forecasting error.}
%From equations~\eqref{eq:bias.2} and~\eqref{eq:mse} combined with results~\eqref{eq:alpha0z.sine},~\eqref{eq:v.sine} and~\eqref{eq:w.sine}, we get:
%\begin{align*}
%\bmu[\ell] &= o(\sigma^2) \\
%\bgamma[\ell,\ell] &= \sigma^2\|\balpha_0^{(\ell)}\|^2 + o(\sigma^2)
%\end{align*}
%Besides, when $\ell=1$ we have:
%\begin{equation*}
%\|\balpha_0^{(1)}\|^2 = \dfrac4{M^2}\sum_{j=1}^J\sum_{m=0}^{M-1}\cos\left(2\pi p_j\dfrac{m}{M}\right)^2 = J\dfrac2M
%\end{equation*}
%Nevertheless, when $\ell\neq 1$, we cannot determine a closed-form expression for $\balpha_0^{(\ell)}$ but we have a recurrence equation:
%\begin{align*}
%\balpha_0^{(\ell)} &= \balpha^{(1)}\tilde\bA_0^{\ell-1} \\
%&= \balpha^{(1)}
%\begin{pmatrix}
%0       & \cdots & 0      & 1      & 0      & \cdots & 0 \\
%\vdots  &        &        & \ddots & 1      & \ddots & \vdots \\
%\vdots  &        &        &        & \ddots & \ddots & 0 \\
%0       & \cdots & \cdots & \cdots & \cdots &    0   & 1 \\
%        &        &        & \balpha_0^{(1)} &        &        &   \\
%        &        &        & \vdots &        &        &   \\
%        &        &        & \balpha_0^{(\ell-1)} &        &        &   
%\end{pmatrix} \\
%&=\balpha_0^{(1)\sharp(\ell -1)} + \sum_{q=1}^{\ell-1}\balpha_0^{(1)}[M-q]\balpha_0^{(\ell-q)}
%\end{align*}
%where
%\begin{equation*}
%\balpha_0^{(1)\sharp(n)} = 
%\begin{pmatrix}
%0 & \cdots & 0 & \balpha_0^{(1)}[0] & \cdots & \balpha_0^{(1)}[M-1-n]
%\end{pmatrix}
%\end{equation*}
%Then:
%\begin{align*}
%\|\balpha_0^{(\ell)}\| &\leq \|\balpha_0^{(1)\sharp(\ell -1)}\| + \sum_{q=1}^{\ell-1}\underbrace{|\balpha_0^{(1)}[M-q]|}_{\leq J\frac2M}\|\balpha_0^{(\ell-q)}\| \\
%&\leq \|\balpha_0^{(1)}\| + \sum_{q=1}^{\ell-1}\dfrac{2J}M\|\balpha_0^{(\ell-q)}\| = \sqrt{\dfrac{2J}M} + \dfrac{2J}M\sum_{q=1}^{\ell-1}\|\balpha_0^{(q)}\|
%\end{align*}
%By induction, we deduce from the previous inequality that
%\[
%\|\balpha_0^{(\ell)}\| \leq \sqrt{\dfrac{2J}M} \left( 1+ \dfrac{2J}M \right)^{\ell-1}
%\]
%Finally:
%\[
%\bv[\ell] \leq \dfrac{2J}{M}\left( 1 + \dfrac{2J}{M}\right)^{2\ell-2}
%\]
%Thus:
%\begin{equation*}
%\bgamma[\ell,\ell] \leq \sigma^2\dfrac{2J}{M}\left( 1 + \dfrac{J}{M}\right)^{2\ell-2} + o(\sigma^2)
%\end{equation*}
%
%Conclusion:
%\begin{align*}
%\bmu[\ell] &= o(\sigma^2) \\
%\left|\bgamma[\ell,\lambda]\right| &\leq \sigma^2\dfrac{2J}{M}\left( 1 + \dfrac{J}{M}\right)^{\ell+\lambda-2} + o(\sigma^2)
%\end{align*}
%
%More generally, the estimated extended signal $\tilde\bx$  takes the following form:
%\begin{equation*}
%\tilde\bx[n] = \bz[n] + \brho[n]\,, \quad \forall\ n\in\{0,\dots,N-1+L\}\ .
%\end{equation*}
%where:
%\begin{equation*}
%\EE\{\brho[n]\} = \left\lbrace
%\begin{array}{ll}
%0\quad &\mbox{if} \quad n\in\{0,\ldots,N-1\}\\
%o(\sigma^2)\quad &\mbox{else,}
%\end{array}
%\right.
%\end{equation*}
%and:
%\begin{equation*}
%\left\lbrace
%\begin{array}{ll}
%\mbox{Cov}\left(\brho[n],\brho[n']\right) = \sigma^2\quad &\mbox{if} \quad (n,n')\in\{0,\ldots,N-1\}^2\\
%\left|\mbox{Cov}\left(\brho[n],\brho[n']\right)\right| \leq \sigma^2\dfrac{2J}{M}\left( 1 + \dfrac{J}{M}\right)^{(n-N)^+ + (n'-N)^+} + o(\sigma^2)\quad &\mbox{if}\ \left\lbrace \begin{array}{l}
%(n\geq N)\cap(n'\geq N-M)\\
%(n'\geq N)\cap(n\geq N-M) \\
%\end{array}
%\right. \\
%\left|\mbox{Cov}\left(\brho[n],\brho[n']\right)\right| \leq \sigma^2\dfrac{2J}{M}\left( 1 + \dfrac{J}{M}\right)^{-(n)^- -(n')^-} + o(\sigma^2)\quad &\mbox{if}\ \left\lbrace \begin{array}{l}
%(n<0)\cap(n'\leq M-1)\\
%(n'<0)\cap(n\leq M-1) \\
%\end{array}
%\right. \\
%\mbox{Cov}\left(\brho[n],\brho[n']\right) = 0\quad &\mbox{else}\ ,
%\end{array}
%\right.
%\end{equation*}
%where $(n)^+$ (resp. $(n)^-$) denotes the positive part (resp. the negative part) of $n$.

Then:
\begin{align*}
\bmu[\ell] &= \EE\{\bh^{(\ell)}\}\bz_{K} + \sigma\EE\{\bh^{(\ell)}\bw_{K}\} + o(\sigma^2) \\
\bgamma[\ell,\ell] &= \EE\{(\bh^{(\ell)}\bz_K)^2\} + 2\sigma\EE\{\bh^{(\ell)}\bz_K\bh^{(\ell)}\bw_K\} + 2\sigma\EE\{\bh^{(\ell)}\bz_K\balpha_0^{(\ell)}\bw_K\} + \sigma^2\EE\{(\bh^{(\ell)}\bw_K)^2\} \\
&\ + 2\sigma^2\balpha_0^{(\ell)}\EE\{\bw_K\bh^{(\ell)}\bw_K\} + \sigma^2\|\balpha_0^{(\ell)}\|^2 - \bmu[\ell]^2 + o(\sigma^2)\ .
\end{align*}

\subsection{Adaptive Harmonic Model}
In this section, the the deterministic part of the observed signal we handle follows the \textit{adaptive harmonic model}, which in its is continuous-time version signal takes the following form:
\begin{equation}
z(t) = \sum_{j=1}^J a_j(t)\cos(2\pi\phi_j(t))\ ,
\end{equation}
where $a_p$ and $\phi'_p$ are smooth function. In other terms, we have:
\begin{align}
|a'_j(t)|<\varepsilon_a,\quad\forall t\in\RR\ ,\\
|\phi''_j(t)|<\varepsilon_\phi,\quad\forall t\in\RR\ ,
\end{align}
for some positive constants $\varepsilon_a$ and $\varepsilon_\phi$.